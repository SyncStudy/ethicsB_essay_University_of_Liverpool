\documentclass{article}
\usepackage[utf8]{inputenc}

\title{Ethical Importance Order of an Engineer}
\usepackage[sort, numbers]{natbib}
\usepackage{graphicx}
\usepackage{geometry}
 \geometry{
 a4paper,
 total={170mm,257mm},
 left=20mm,
 top=20mm,
 }
\begin{document}

\maketitle

\section{Scenario}
\subsection{Facing Airbus's competition}
\hspace{\parindent}
There are two main aeroplane companies in the world, which are Airbus and Boeing. In 2010, Airbus introduced the new A320, named as A320neo and it integrated a new turbofan engine, which makes the plane 15\% more fuel-efficient \cite{Therealr56:online}. Additionally, pilots only need a minimum of additional training for this new model. These two factors make A320neo a top hit in the market, and in 2011, the Sales for Airbus is nearly ten times that in Boeing \cite{AirbusA376:online}. Boeing faced the competition from Airbus and sought to make improvements to their competing plane Boering 737, for higher fuel efficiency with minimal additional training. However, since the landing gear for 737 is shorter than that in A320, there is little space to extend for a larger engine for 737. Due to the market pressure for the new version, Boeing introduces the new engine slightly above the wings to extend the space. However, this will cause the nose of the plane to move upward, which increases the risk of stall \cite{Therealr56:online}. Based on that, Boeing introduces Maneuvering Characteristics Augmentation System (MCAS) to automatically push the nose down. Like Airbus, Boeing introduces a 2-hour iPad course to quickest train pilots for a new plane (737 Max). Boeing 737 Max becomes a hit in the market.
\subsection{After first crash}
On October 29, 2018, Lion Air Flight 610 took off from Jakarta, Indonesia in 737 Max. During the flight, MCAS got incorrect sensor data which keeps pushing the nose down while pilots tried to pull planes up \cite{Therealr56:online}. 12 minutes later, the plane crashed in Java sea, killing 189 people. Later, the despair relatives questioned whether the 737 MAX, needed to be grounded over safety concerns, but the Indonesian government said operators need to do nothing more than carry out additional checks \cite{Families79:online}. Boeing makes settlements to victims and took the action for public relationship \cite{Ethiopia52:online}. Half a year later, in March of 2019, the second Boeing 737 Max crashed due to similar problems, killing 157. The investigators said pilots follow all procedures instructed by Boeing, and all Boeing 737 planes have been grounded, followed by serious aeroplane investigation.

 
\section{Dilemma}
\subsection{Facing Airbus's competition}
Facing Airbus's new model and the difficulties in making improvements based on the original plane design, which is much more expensive, Boeing has two options. One is to redesign the plane comprehensively to make the plane capable of large engines, and the other is to add patches to the current plane, reducing the time to market and cost of additional training to the minimal. The newly designed plane essentially have a higher degree of safety since it incorporates a higher level of integration. However, it costs much for Boeing from designing, implementing, testing, and training. Additionally, the new plane will require authentication from air regulatory institutions, delaying the time for the market, which makes Airbus much profitable, in the short term. On the other hand, adding patches based on the original version to incorporate heavy engines reduces the system integration level, which potentially increases the possibilities for risk. However, Boeing significantly reduces the cost of designing new planes, requiring less time for authentication from authorities, and training from airline companies.
\subsection{After first crash}
After the first crash, Boeing had two ways to go. The first way was to ground 737 Max, and tested until the planes are sure to be safe with MCAS. Contrastly, Boeing could also focus on the public relations staff, accusing the pilots of not behaving properly as the instructions, and eliminating the effects of this accident to a minimum through offering high settlement fee. Grounding planes and redo the safety test is safe but expensive since every second where 737 Max is grounded makes Airbus earning much more money in the market. Additionally, the grounded safety check will even influence Boeing's other versions of planes, since the market may be panic about the grounded result and it will significantly influence the stock price for Boeing. On the other hand, if Boeing focuses on public relations, it will eliminate its influence on the market to a minimum and effectively compete with Airbus. Throughout reasonable settlement, and accusing improper behaviour from pilots are common methods to avoid responsibility and earn money in the market within a short period \cite{American37:online}. However, without a comprehensive safety improvement, and accusing pilots is never a proper method to value passengers' safety and it will lose the customer's trust if the 737 Max crashes again in the future.
\section{What could I do?}
\subsection{Facing Airbus's competition}

\subsubsection{Option 1 from managers in Boeing}
The first and foremost option is to keep in mind that as an aeroplane company, the customer's safety is the root for making profits on the market. Without the trust of the customer, there is no way to make profits in the long term. Therefore, the leaders and managers should be told the staffs about the position of the company as prioritising the customer's welfare and regarding the long-term profits making abilities rather the short term competition with Airbus.
\subsubsection{Option 2 from engineers in Boeing}
Facing the new generation of engine and the 737's build-in difficulties in integration larger engine, engineers should try improving the level of integration and undertaken the sufficient number of critical safety examinations to make the MCAS system well integrated into the system. Rather chasing markets with Airbus, ignoring the potential for safety risks.
\subsubsection{Option 3 from training staff}
Even though the similarities between 737 and 737 Max are the selling point of this plane, differences should be emphasized for safety considerations. The training staff should introduce the MCAS to pilots regarding its principles, functionalities for preventing stalling, and interfaces to control this system. Rather ignoring that for convenience. Essentially, every new feature introduced in 737 Max should be well-trained to save time when an emergency happens. 
\subsection{After the first crash}
\subsubsection{Option 1 from managers in Boeing}
They were expected to arrange the emergency meeting regarding this crash. They should emphasize the safety of the public, rather how to eliminate the cost for the company in the short run. Additionally, they should do a comprehensive survey for understanding the reasons behind the crash, and ask engineers to examine whether build-in flight version errors are leading the crash. Finally, they should re-emphasize the importance of safety as the key to survival in the area of the flight plane domain.
\subsubsection{Option 2 for engineers in Boeing}
There were two things to be done, which are evaluation and implementation. Firstly, engineers should evaluate the reason behind the crash based on a comprehensive survey. Based on the BBC news, the main reasons were based on the erroneous data read from the plane nose, which misguided the MCAS to push the nose down \cite{Therealr56:online}. After the survey, engineers should implement the updated software and add new sensor to guard against erroneous data.

\subsubsection{Option 3 for training staff and pilots for 737 Max}
New guidelines for pilots should be adopted. According to Guardian, when the emergency happened, the pilots were finding the emergence handbook, significantly increasing the possibilities for the crash \cite{LionAirp71:online}. All pilots from 737 should learn to adapt to the MCAS system in 737 Max, and the training staff should lay greater emphasis on the MCAS to ensure that. 
\section{Discussion – logic for best option} 
There are four engineering ethical principles involved in Boeing's decision-making process, which are Accuracy and rigour, Honesty and integrity, Respect for life, law and the public good, Responsible leadership: listening and informing, according to Royal Academy of Engineering’s Statement of Ethical Principles \cite{statemen88:online}, illustrated as followed. 
\subsection{Accuracy and rigour}
Engineering professionals have a duty to uphold the highest
standards of professional conduct including openness, fairness,
honesty and integrity \cite{statemen88:online}. They not only act in a reliable and trustworthy manner but also avoid deception and take steps to prevent professional misconduct \cite{statemen88:online}. In this case, Boeing should not be pushed by the market, and introduce the plane, which did not pass the sensor data reading test, and system intensity test, directly leading to the crashes. I think reliable is the moral of engineers, which means the accuracy and rigour are the base of engineering development, not the market profit in the short term. Only guaranteed with the accuracy and rigour, can the company acquires the abilities to make profits in the long run. Additionally, regarding taking steps to avoid professional misconduct, Boeing should not accuse the pilots to be guilty of the crash but deeply reflect what Boeing has done wrong to solve the problem in the future. Every problem in the world is originated to lack of abilities, thus accusing others is of no use for solving the problem, but reflecting and improving abilities can earn and re-earn the trust of customers. Therefore, they should not focus on the market competition with Airbus, but focusing on the accuracy and rigour of their product. Also, they should not blame pilots for the crash but do reflections deeply within the company to reduce cases as this happens in the future.  Based on reliable and safe aeroplanes, originated from accuracy and rigour, Boeing can make profits in the long run. 
\subsection{Honesty and integrity}
The definition of honesty and integrity is as follows, 
Engineering professionals have a duty to obey all applicable laws
and regulations and give due weight to facts, published standards
and guidance and the wider public interest \cite{statemen88:online}. 
There are three factors to be considered in this case, illustrated as follows.
Firstly, Boeing should hold paramount the health and safety of others and draw
attention to hazards \cite{statemen88:online}. During the new plane induction training, Boeing should have introduced the MCAS system and its functionalities for anti-stalling. Additionally, how to manually turn on and off this system to effectively interact with the system. The second principle in honesty and integrity part is to maximise the public good and minimise adverse effects correspondingly. According to consequentialism, a morally right act is one that will produce the greatest outcome, or consequence for the greatest people \cite{Conseque99:online}. If the plane crashed, Boeing company will kill hundreds of people, resulting in unrecoverable painfulness for their relatives and families. Therefore, Boeing should keep the honesty and integrity for the seek of a public good. Last but not least, Boeing should uphold the reputation and standing of the profession \cite{IEEEIEEE41:online}. Since the reputation of a company determines the corresponding market response. If the company want to make money in the long run, only proficient abilities in their domain are not enough, the well-structured market response is essential. Boring should have considered what crashing accidents will do to their public reputation, and they should go out of their way to ensure it protects the reputation throughout honesty and integrity.
\subsection{Respect for life, law and the public good}
The definition of respect for life law and the public good in the engineering domain is as follows. Engineering professionals have a duty to acquire and use wisely
the understanding, knowledge and skills needed to perform their
role \cite{Conseque99:online}. There are three aspects of this ethical topic, to be illustrated as follows. They should always act with care, Boeing should always keep in mind that they are undertaking hundreds of lives on broad. The public good should be emphasized compared with the market profits competing Airbus. Additionally, Boeing should perform services only in areas in which they are currently competent or under competent supervision.  This aspect is mainly about the air regulators, they should not adopt the streamlined flying certificates to new planes but carefully check the components, system integration level to fully achieve the high standard of safety guaranteeing the public good. Finally, Boeing engineers should identify, evaluate, quantify, mitigate and manage risks not only from the risk of stalling but also the likelihood where it will push the nose too downward to make the plane crash. All these three aspects are for the protection for public goods, life, and law. Respect is a bi-directional behaviour, if you respect others, others will respect you. Just as if Boeing respect the life of the public, the public will respect Boeing's work through flying with Boeing more in the future, and vice versa.
\subsection{Responsible leadership: listening and informing}
The definition for the responsible leadership: listening and informing is as follows. Engineering professionals have a duty to abide by and promote
high standards of leadership and communication \cite{statemen88:online}. In this case, there are three aspects involved to be illustrated as follows.
They should be aware of the issues that engineering and technology raise for
society, and listen to the aspirations and concerns of others \cite{statemen88:online}. Additionally, they should promote public awareness and understanding the seriousness of the crash. In this way, customers have known that Boeing is indeed treasuring the public goods, and that is the base of the trust. Eventually, Boeing should be objective and truthful in any statement made in their professional capacity. That mainly refers to the statement from the CEO of Boeing, he said that Boeing's new version 737 will be the safest plane in the world and it will be published within three months \cite{6BoeingC1:online}. The safest plane should take a longer time to construct in common sense, and hopefully, Boeing will earn and re-earn the trust from customers in the future, based on the current capabilities for production. Therefore, Boeing should inform their pilots about the software system, MCAS, they should let the public aware the impact of this system, and finally, they should inform the public within its capacity of production in order to follow the responsible leadership: listening and informing.
\section{Conclusion}
As a world-wide aeroplane company, they put the safety priority, rather the competition with Airbus for the market. Just like us as an engineer, we should put safety (health), as the top priority, then our abilities to make future development. Throughout the practices, we should obey the accuracy and rigour, honesty and integrity, respect for life, law and the public good, and responsible leadership: listening and informing. Finally, it is the issues related to money and the market. That is the ethical importance order of an engineer.
\bibliographystyle{IEEEtran}
\bibliography{references}
\end{document}
